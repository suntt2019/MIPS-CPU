\documentclass[main.tex]{subfiles}

\begin{document}

\section{模块设计}

\paragraph{增量介绍说明}
鉴于Project2基于Project1开发,大部分模块与Project1一致,这里只介绍进行改动或新增加的模块。与Project1一致的模块详见报告第一部分。

\subsection{IFU}
\subsubsection{基本描述}
IFU完成程序计数器$PC$的维护、以及指令存储器$IM$的读取。
\paragraph{程序计数器$PC$的维护}
程序计数器$PC$的维护工作,即是在每次时钟信号$clk$出现上升沿时,根据输入的控制信号$NPCSel$,使用对应的方式生成新的PC值。

支持四种新$PC$生成方式。$PC+4$模式为大部分指令服务,相对跳转模式为$beq$等指令服务,绝对跳转模式 为$j$等指令服务,寄存器模式为$jr$等指令服务。

\paragraph{指令存储器$IM$的读取}
程序计数器$PC$的值即是指令存储器$IM$的地址,该模块时刻读出$PC$指向$IM$中的值。

\subsubsection{模块接口}
\begin{center}
    \captionof{table}{IFU模块接口}
    \begin{tabular}[]{c c c l}
        \toprule
        信号名称 & 方向 & 位宽 & 描述 \\
        \midrule
        $clk$ & input & 1bit & 时钟信号。\\
        $reset$ & input & 1bit & \makecell[lt]{
            复位信号。\\
             0:复位。\\
             1:无效。
        } \\
        $PCWr$ & input & 1bit & (新增)高有效的$PC$写使能信号。\\
        $NPCSel$ & input & 2bit & \makecell[lt]{
            新$PC$值生成方式选择的控制信号。\\
            NPC\_SEL\_PC\_ADD\_4(0b00):使用$PC+4$模式。\\
            NPC\_SEL\_BEQ\_JMP(0b01):使用相对跳转模式。\\
            NPC\_SEL\_J\_JMP(0b10):使用绝对跳转模式。\\
            NPC\_SEL\_REG\_JMP(0b11):使用寄存器跳转模式。
        } \\
        \midrule
        $StoredInstr$ & input & 32bit & (新增)$IR$寄存器中存储的指令。 \\
        $regPC$ & input & 32bit & 寄存器跳转到的地址。\\
        $Instr$ & output & 32bit & 32bit的$MIPS$指令。\\
        $pc$ & output & 32bit & 当前$PC$的值。 \\
        \bottomrule
    \end{tabular}
\end{center}

\clearpage
\subsubsection{功能定义}
\begin{center}
    \captionof{table}{IFU功能定义}
    \begin{tabular}{c c l}
        \toprule
        序号 & 功能名称 & 功能描述 \\
        \midrule
        1 & 复位 & $reset \land clk\uparrow \Rightarrow  PC \leftarrow 0x000003000$ \\
        2 & 取指令 & $Instr \leftarrow IM[PC] $ \\
        3.1 & $PC+4$模式 & $NPCSel == 0b00 \Rightarrow PC \leftarrow PC + 4 $ \\
        3.2 & 相对跳转 & (更新)$NPCSel == 0b01 \Rightarrow PC \leftarrow PC + StoredInstr_{15..0}\ ||\ 0^2$ \\
        3.3 & 绝对跳转 & (更新)$NPCSel == 0b10 \Rightarrow PC \leftarrow PC_{31..28}\ ||\ StoredInstr_{25..0}\ ||\ 0^2 $ \\
        3.4 & 寄存器跳转 & $NPCSel == 0b11 \Rightarrow PC \leftarrow regPC $ \\
        \bottomrule
    \end{tabular}
\end{center}

\subsubsection{模块实现}

IFU模块中,每次$clk$出现上升沿时,根据$NPCSel$选择模式,计算新的$PC$值并写入$PC$寄存器。

$PC$值作为指令存储器$IM$的地址,固定将对应数据输出到$Instr$。

\subsubsection{变化总结}
IFU模块的变化主要在于两点:

\paragraph{使用缓存后的$StoredInstr$}
Project2的IFU使用$StoredInstr$代替原来$Instr$作为跳转数据来源。这是由于多周期当前执行指令固定存储在$IR$中的性质带来的,原本每条指令执行在一个周期中,可以直接使用$Instr$而多周期中这样会读成下一条指令,故此读取缓存的$StoredInstr$。

\paragraph{使用$PC$而非$PC+4$}
Project2的IFU使用$PC$而非$PC+4$作为跳转运算的基础。这也是由多周期的性质带来的,多周期处理器在取指阶段固定将$PC$变为$PC+4$,因此若之后再进行跳转,当时的$PC$已经相当于是单周期时的$PC+4$了,故此改变IFU保证执行效果统一。

\clearpage \subsection{BAC}
\subsubsection{基本描述}
BAC完成访存主存时,字节与字之间的转换。具体包括地址转换、读取内容转换、写入内容转换。

BAC模块为Project2新增模块。

\paragraph{地址转换}
按字节访存时,BAC将访问字节时的地址完成4的倍数对齐,并保证访存的字范围包含字节范围。

\paragraph{读取内容转换}
按字节读取时,BAC将读到的字节固定放到字的最低位,并进行符号拓展。

\paragraph{读取内容转换}
按字节写入时,BAC先从主存读取对应的字,将待写入的字节放到这个字的对应位置,再将字写回主存。


\subsubsection{模块接口}
\begin{center}
    \captionof{table}{BAC模块接口}
    \begin{tabular}[]{c c c l}
        \toprule
        信号名称 & 方向 & 位宽 & 描述 \\
        \midrule
        $BACOp$ & input & 1bit & \makecell[lt]{
            选择访存模式。\\
            BAC\_OP\_WORD(0):使用字访存模式。\\
            BAC\_OP\_BYTE(1):使用字节访存模式。\\
        } \\
        \midrule
        $Ain$ & input & 32bit & 输入的地址。 \\
        $Din1$ & input & 32bit & 待写入的数据。 \\
        $Din2$ & input & 32bit & 从$dm\_1k$读取的数据。\\
        $Aout$ & output & 32bit & 输出给$dm\_1k$的地址。\\
        $Dout1$ & output & 32bit & 输出给$dm\_1k$的数据。 \\
        $Dout2$ & output & 32bit & 读取的结果。 \\
        \bottomrule
    \end{tabular}
\end{center}

\subsubsection{功能定义}
\begin{center}
    \captionof{table}{BAC功能定义}
    \begin{tabular}{c c l}
        \toprule
        序号 & 功能名称 & 功能描述 \\
        \midrule
        1 & 生成字地址 & $BACOp = BYTE \Rightarrow Aout \leftarrow Ain_{31..2}\ ||\ 0^2$ \\
        2 & 获取字中位置 & $d \leftarrow Ain_{1..0} << 3$ \\
        3 & 字节读取 & \makecell[lt]{
                        $byte2 \leftarrow (Din2>>(d))_{3..0}$ \\
                        $BACOp = BYTE \Rightarrow Dout2 \leftarrow byte2[7]^{24}\ ||\ byte2$
        } \\
        4 & 字节写入 & \makecell[lt]{
                        $BACOp = BYTE$ \\
                        $\ \ \Rightarrow Dout1 \leftarrow Din2_{31..(d+8)}\ ||\ Din1{7..0}\ ||\ Din2_{d-1..0}$ \\
        } \\
        5 & 字访存 & $BACOp = WORD \Rightarrow Aout\leftarrow Ain, Dout1\leftarrow Din1, Dout2\leftarrow Din2$ \\
        \bottomrule
    \end{tabular}
\end{center}

\subsubsection{模块实现}
BAC为组合电路,当输入发生变化时,根据$BACOp$对应进行字节或字访存的数据转换。


\clearpage \subsection{Controller}
\subsubsection{基本描述}
Controller完成指令的种类判断,状态跳转,并输出对应的控制信号。

\paragraph{指令种类判断}
根据输入的$opcode$和$funct$信号判断指令种类。

\paragraph{状态跳转}
根据当前指令种类、标志位信号$StoredNFlag$跳转到新的状态。

\paragraph{产生控制信号}
根据判断出的状态,选取对应的控制信号输出。

\subsubsection{模块接口}
\begin{center}
    \captionof{table}{controller模块接口}
    \begin{longtable}[]{c c c l}
        \toprule
        信号名称 & 方向 & 位宽 & 描述 \\
        \midrule
        $clk$ & input & 1bit & 时钟信号。\\
        $reset$ & input & 1bit & \makecell[lt]{
            复位信号。\\
             0:复位。\\
             1:无效。
        } \\
        \midrule
        $opcode$ & input & 6bit & $MIPS$指令中的高6位,用于区分指令种类。\\
        $funct$ & input & 6bit & $MIPS$指令中的低6位,R类型指令中用于区分指令种类。 \\
        $NFlag$ & input & 32bit & 新计算出的标志位。 \\
        \midrule
        $RegDst$ & output & 2bit & \makecell[lt]{
            写回的寄存器编号选择信号。\\
             REGDST\_RT(0b00):写回到$rt$。 \\
             REGDST\_RD(0b01):写回到$rd$。\\
             REGDST\_RET(0b10):写回到返回值寄存器($\$31$)。
        } \\
        $ALUSrc$ & output & 1bit & \makecell[lt]{
            $ALU$第二个运算数的输入选择信号。\\
             ALUSRC\_B(0):选择$B$。 \\
             ALUSRC\_EXT(1):选择$Ext$。
        } \\
        $Mem2Reg$ & output & 2bit & \makecell[lt]{
            写回内容的选择信号。\\
             MEM2REG\_ALU(0b00):写回$ALU$计算结果。 \\
             MEM2REG\_RAM(0b01):写回$dm\_1k$中读出值。\\
             MEM2REG\_RET(0b10):(更新)写回$PC$。\\
        } \\
        \midrule
        $PCWr$ & output & 1bit & (新增)$PC$的高有效写使能信号。\\
        $IRWr$ & output & 1bit & (新增)$IR$的高有效写使能信号。\\
        $RegWr$ & output & 1bit & 寄存器的高有效写使能信号。\\
        $MemWr$ & output & 1bit & 内存的高有效写使能信号。\\
        $ALUOWr$ & output & 1bit & (新增)$ALU$输出的高有效写使能信号。\\
        \midrule
        $BACOp$ & output & 1bit & 模块$BAC$控制信号,详见其说明。\\
        $NPCSel$ & output & 2bit & 模块$IFU$控制信号,详见其说明。\\
        $EXTOP$ & output & 2bit & 模块$EXT$控制信号,详见其说明。\\
        $FlagOP$ & output & 2bit & 模块$GPR$控制信号,详见其说明。\\
        $ALUOp$ & output & 3bit & 模块$ALU$控制信号,详见其说明。\\
        \bottomrule
    \end{longtable}
\end{center}

\subsubsection{功能定义}

多周期处理器存在多个状态,控制器的本质是一个有限状态机。这里我们给出指令类型、状态、标志位和控制信号之间的关系。

\paragraph{状态按执行步骤分类}
处理器执行指令包括:取指、译码与寄存器读取、计算、访存主存、写回,这五个步骤。而对于跳转指令,还存在跳转步骤。故此一共六个执行步骤。

根据这6个执行步骤,我们可以将状态进行分类,五个步骤分别记为$S1$至$S5$,而跳转步骤记为$S6$。

特别地,一些需要进行判断的特别步骤通常在计算($S3$)之后,为了方便表示,我也将其归入$S3$,不过其性质与计算并不一致。

\paragraph{指令类型决定大部分状态转移}
大部分的状态转移具有很强的确定性,每条指令存在一个状态循环节,每走完一次这个循环节即执行了一次这条指令。

不同指令的状态循环节有所不同,但可根据状态按执行步骤的分类归纳如下表的“指令类型-状态循环节”对应关系。

\begin{center}
    \captionof{table}{各指令的状态循环节}
    \begin{tabular}{ccccccc}
        \toprule
        指令类型 & S1 & S2 & S3 & S4 & S5 & S6 \\
        \midrule
            ADDU    &   S1 & S2 & S3\_EXE\_ADD     &-              &S5\_ALU\_R\_FD   & - \\
            SUBU    &   S1 & S2 & S3\_EXE\_SUB     &-              &S5\_ALU\_R\_FD   & - \\
            ORI     &   S1 & S2 & S3\_EXEI\_OR     &-              &S5\_ALU\_I\_FD   & - \\
            LW      &   S1 & S2 & S3\_EXEI\_ADD    &S4\_RD\_WORD    &S5\_DM\_WORD    & - \\
            SW      &   S1 & S2 & S3\_EXEI\_ADD    &S4\_WR\_WORD    & -             & - \\
            BEQ     &   S1 & S2 & S3\_EXE\_SUB     &S3\_BR\_BEQ     & -             &S6\_BEQ  \\
            J       &   S1 & S2 & -               & -             & -             &S6\_J  \\
            LUI     &   S1 & S2 & S3\_EXEI\_LUI    &-              &S5\_ALU\_I\_FD   & - \\
            ADDI    &   S1 & S2 & S3\_EXEI\_ADD,   &S3\_CK\_OF      &S5\_ALU\_I\_FD   & - \\
            ADDIU   &   S1 & S2 & S3\_EXEI\_ADD    &-              &S5\_ALU\_I\_FD   & - \\
            SLT     &   S1 & S2 & S3\_EXE\_SLT     &-              &S5\_ALU\_R\_FD   & - \\
            NOP     &   S1 & S2 &-                & -             & -             & -  \\
            LB      &   S1 & S2 & S3\_EXEI\_ADD,   &S4\_RD\_BYTE    & S5\_DM\_WORD   &-  \\
            SB      &   S1 & S2 & S3\_EXEI\_ADD,   &S4\_WR\_BYTE    & -             & -  \\
            JAL     &   S1 & S2 &-                & -             &S5\_RET        &S6\_J  \\
            JR      &   S1 & S2 &-                & -             & -             &S6\_REG  \\
        \bottomrule
    \end{tabular}
\end{center}

\paragraph{标志位决定少部分状态转移}
由于少数指令需要执行到指定周期才能确定后续跳转的状态,故此标志位会决定少数的状态转移,相当于是系统中的特例,如下表所示。

\begin{center}
    \captionof{table}{标志位决定的状态转移}
    \begin{tabular}{cccc}
        \toprule
        源状态 & 标志位名称 & 为1时到达 & 为0时到达 \\
        \midrule
            S3\_BR\_BEQ & zero & S6\_BEQ & S1 \\
            S3\_CK\_OF & overflow & S5\_ALU\_I\_FS & S5\_ALU\_I\_FD \\
        \bottomrule
    \end{tabular}
\end{center}

\paragraph{状态决定控制信号}
对于每个指定的状态,controller的输出是固定的,下面给出状态-控制信号对应表(见下一页,省略控制信号前缀)。

\paragraph{功能总结}
由上述的分别描述,我们可以将功能总结为如下定义。

\begin{center}
    \captionof{table}{controller功能定义}
    \begin{tabular}{c c l}
        \toprule
        序号 & 功能名称 & 功能描述 \\
        \midrule
        1 & 复位 &  $reset \Rightarrow status \leftarrow S2$\\
        2 & 指令种类判断 &  由$opcode$和$funct$判断指令类型$InstrID$\\
        3 & 下一状态生成 &  \makecell[lt]{
                             由指令类型$InstrID$、当前状态$status$\\
                             \ \ 和标志位$StoredNFlag$,生成下一状态
                            }\\
        4 & 生成控制信号 &  由状态$status$确定输出信号 \\
        \bottomrule
    \end{tabular}
\end{center}

\subsubsection{模块实现}
\paragraph{整体概述}
controller模块为有限状态机,对$clk$和$reset$的上升沿敏感。当$clk$出现上升沿时,首先判断指令种类,再在状态循环节中找到下一状态,若此状态需要经过标志位判断跳转则再次转移状态,最后将与状态对应的控制信号输出。

\paragraph{特别的有限状态机策略}
本模块有限状态机的实现方法与课内所讲有所不同。课内的实现方式优先根据当前状态分类,再判断指令类型、标志位决定下一状态;而本模块优先根据指令种类分类,再在少数状态下根据标志位分两步转移到下一状态。

\paragraph{自动化的状态转移生成功能}
值得一提的是,本模块在复位的时候,其实会进行每类指令状态转移$next$数组的生成(由于与外部特性无关,仅体现在实现部分)。

初始化时,controller会将预设好的每类指令的状态循环节$steps$变为状态转移$next$数组。这样一来,开发者一端可以清晰明了地看到每条指令的状态循环节;而程序运行时,只需要将当前状态作为下标输入给$next$数组,就能轻松得到循环节中的下一个状态了。

这样的设计为修改状态循环节提供了极大的便利,针对添加指令、调试等工作十分方便。但同时,这样的结构若直接写入硬件也会导致大量的寄存器低频使用,十分浪费。个人认为这是此方法的优劣所在。

\clearpage

\Rotatebox{90}{%
\begin{tabular}[]{cccccccccccccc}
    \toprule
    状态名 & PCWr & IRWr & RegWr & MemWr & ALUOWr & ALUSrc & RegDst & Mem2Reg & BACOp & NPCSel & EXTOp & FlagOp & ALUOp \\
    \midrule
     S1  & EN  & EN  & DIS  & DIS  & DIS & ZZ  & ZZ  & ZZ & ZZ  & ADD\_4  & ZZ  & ZZ  & ZZ \\
    S2  & DIS  & DIS  & DIS  & DIS  & DIS & ZZ  & ZZ  & ZZ & ZZ  & ZZ  & ZZ  & ZZ  & ZZ \\
    S3\_EXE\_ADD  & DIS  & DIS  & DIS  & DIS  & EN & B  & ZZ  & ZZ & ZZ  & ZZ  & ZZ  & ZZ  & ADD \\
    S3\_EXE\_SUB  & DIS  & DIS  & DIS  & DIS  & EN & B  & ZZ  & ZZ & ZZ  & ZZ  & ZZ  & ZZ  & SUB \\
    S3\_EXE\_SLT  & DIS  & DIS  & DIS  & DIS  & EN & B  & ZZ  & ZZ & ZZ  & ZZ  & ZZ  & ZZ  & LESS \\
    S3\_EXEI\_ADD  & DIS  & DIS  & DIS  & DIS  & EN & EXT  & ZZ  & ZZ & ZZ  & ZZ  & SIGN  & ZZ  & ADD \\
    S3\_EXEI\_OR  & DIS  & DIS  & DIS  & DIS  & EN & EXT  & ZZ  & ZZ & ZZ  & ZZ  & ZERO  & ZZ  & OR \\
    S3\_EXEI\_LUI  & DIS  & DIS  & DIS  & DIS  & EN & EXT  & ZZ  & ZZ & ZZ  & ZZ  & LUI  & ZZ  & OR \\
    S3\_EXE\_LTZ  & DIS  & DIS  & DIS  & DIS  & EN & B  & ZZ  & ZZ & ZZ  & ZZ  & ZZ  & ZZ  & LESS \\
    S3\_BR\_BEQ  & DIS  & DIS  & DIS  & DIS  & DIS & ZZ  & ZZ  & ZZ & ZZ  & ZZ  & ZZ  & ZZ  & ZZ \\
    S3\_BR\_BLTZAL  & DIS  & DIS  & DIS  & DIS  & DIS & ZZ  & ZZ  & ZZ & ZZ  & ZZ  & ZZ  & ZZ  & ZZ \\
    S3\_CK\_OF  & DIS  & DIS  & DIS  & DIS  & DIS & ZZ  & ZZ  & ZZ & ZZ  & ZZ  & ZZ  & ZZ  & ZZ \\
    S4\_RD\_WORD  & DIS  & DIS  & DIS  & DIS  & DIS & ZZ  & ZZ  & ZZ & WORD  & ZZ  & ZZ  & ZZ  & ZZ \\
    S4\_RD\_BYTE  & DIS  & DIS  & DIS  & DIS  & DIS & ZZ  & ZZ  & ZZ & BYTE  & ZZ  & ZZ  & ZZ  & ZZ \\
    S4\_WR\_WORD  & DIS  & DIS  & DIS  & EN  & DIS & ZZ  & ZZ  & ZZ & WORD  & ZZ  & ZZ  & ZZ  & ZZ \\
    S4\_WR\_BYTE  & DIS  & DIS  & DIS  & EN  & DIS & ZZ  & ZZ  & ZZ & BYTE  & ZZ  & ZZ  & ZZ  & ZZ \\
    S5\_DM\_WORD  & DIS  & DIS  & EN  & DIS  & DIS & ZZ  & RT  & RAM & WORD  & ZZ  & ZZ  & DIS  & ZZ \\
    S5\_DM\_BYTE  & DIS  & DIS  & EN  & DIS  & DIS & ZZ  & RT  & RAM & BYTE  & ZZ  & ZZ  & DIS  & ZZ \\
    S5\_ALU\_I\_FD  & DIS  & DIS  & EN  & DIS  & DIS & ZZ  & RT  & ALU & WORD  & ZZ  & ZZ  & DIS  & ZZ \\
    S5\_ALU\_I\_FS  & DIS  & DIS  & EN  & DIS  & DIS & ZZ  & RT  & ALU & WORD  & ZZ  & ZZ  & SET  & ZZ \\
    S5\_ALU\_R\_FD  & DIS  & DIS  & EN  & DIS  & DIS & ZZ  & RD  & ALU & WORD  & ZZ  & ZZ  & DIS  & ZZ \\
    S5\_RET  & DIS  & DIS  & EN  & DIS  & DIS & ZZ  & RET  & RET & WORD  & ZZ  & ZZ  & DIS  & ZZ \\
    S6\_BEQ  & EN  & DIS  & DIS  & DIS  & DIS & ZZ  & ZZ  & ZZ & ZZ  & JMP  & ZZ  & ZZ  & ZZ \\
    S6\_J  & EN  & DIS  & DIS  & DIS  & DIS & ZZ  & ZZ  & ZZ & ZZ  & JMP  & ZZ  & ZZ  & ZZ \\
    S6\_REG  & EN  & DIS  & DIS  & DIS  & DIS & ZZ  & ZZ  & ZZ & ZZ  & JMP  & ZZ  & ZZ  & ZZ \\
    \bottomrule
\end{tabular}
}%

\clearpage

\end{document}
