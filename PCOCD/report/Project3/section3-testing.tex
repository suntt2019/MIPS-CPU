\documentclass[main.tex]{subfiles}

\begin{document}
\section{程序运行测试}

\subsection{测试程序}
我按照老师要求,开发以下程序并进行测试:

主程序加载到0x0000\_3000,内容如下:
\inputminted[linenos]{gas}{Project3/p3-test-main-commented.asm}

中断服务程序加载到0x0000\_4180,内容如下:
\inputminted[linenos]{gas}{Project3/p3-test-int-commented.asm}

\clearpage

\subsection{测试方法与结果}

\paragraph{测试方法}
在$IFU$中修改$PC$处增加系统任务$\$display$代码,在$PC$到达特定值时输出log。使用$ModelSim$对系统进行仿真测试,观察log结果是否与预期一致以测试系统。

添加的log功能如下表所示。

\begin{center}
    \captionof{table}{Project3 $PC$监测点}
    \begin{longtable}[]{c c c c}
        \toprule
        监测地址 & 地址意义 & 输出内容 \\
        \midrule
        0x4180 & 中断服务程序开始运行 & Start int \\
        0x4190 & 置数变化,更新数值 & New IN32 \\
        0x4198 & 置数不变,数值加一 & Add 1 \\
        0x41a4 & 重置计数器,重新计数 & Preset \\
        \bottomrule
    \end{longtable}
\end{center}

\paragraph{测试结果}

\subparagraph{log结果} log中第一次出现 "Start int", "New IN32", "Preset" 序列,之后持续出现 "Start int", "Add 1", "Preset" 序列。若修改输入内容(置数内容),则再出现一次"Start int", "New IN32", "Preset"序列。可见硬件中断处理正常,中断服务程序正常运行。

\subparagraph{输出结果} 将$DOut$值输出,可见其规律性增长,每次增长间隔均匀。尝试修改测试程序中$\$10$存储的值,增长频率显著发生变化,可见硬件中断请求产生正常,计时器正常工作。

\subparagraph{整体总结} 测试结果与预期一致,验证了系统的工作正确性。

\end{document}
