\documentclass[main.tex]{subfiles}
\begin{document}

\section{单元测试}
本次课设开发过程中,让我十分印象深刻的一点就是单元测试了。之前在开发web后端的时候,由于需要考虑到用户各种不同的情况,以及网络上存在的安全风险,十分需要单元测试。单元测试通过将测试的粒度减小的方式,在对于需要测试面全面的情况大幅降低了工作量,并提高了覆盖范围。这个思路也延续到我进行课设开发中。

但课设开发的性质与之前我进行的web后端开发并不一致。课设追求的检查其实主要就是测试程序运行正确,对不同情况的要求并不多。因此我开发单元测试提高覆盖率就没有太大的价值。相反,我还耗费了大量的时间精力在开发单元测试上。再有,最后上到FPGA上的时候,由于单元测试代码质量不是很好,而且较为复杂,甚至导致ISE出现了内部错误,反倒是引起了更多麻烦。

这次开发经历让我体会到技术应用到不同场合的收效很可能十分不同。例如单元测试大幅度提高了web后端的测试效率,却大量浪费了我开发课设的时间。这点让我对单元测试、以及其他开发技术都有了更新、更立体的认识。

\section{鸣谢}
在此特意感谢教授我课程知识、帮助我解决技术问题的朱文军老师。前面提到出现ISE内部错误,当时让我完全不知所措。但我去找到朱老师的时候,老师还是耐心地帮我分析问题、解决问题,还提供他的电脑作为对照实验参照。此后还有许多问题都请教了老师,老师也不厌其烦地帮助了我。在此为老师专业的技术和教导学生的精神感到无比的钦佩,以及对老师对我的帮助十分感谢。


\end{document}