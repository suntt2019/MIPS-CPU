\documentclass[main.tex]{subfiles}
\begin{document}
\section{总结与收获}

\subsection{开发与调试过程}

\paragraph{认识CPU从黑盒到多模块机器}
其实我个人一直认为这项大作业是一个很神奇的过程。从接触计算机,到学习高级语言,再到学习汇编语言,虽然对计算机的了解越来越深入,我却一直将CPU整个视作一个黑盒。而现在通过几个组合电路以及一些存储器就能组合出这样CPU,个人感觉是个很神奇的事情。

不过这次作业让我对CPU的体系结构有了更加深刻的认识,之前上课虽然也听懂了大部分的内容,但细节还是要亲自实践才能熟悉。虽然由于大部分用途下还是要将CPU当作黑盒便于思考,但我通过实践加深理解,可以尝试将其看作多模块组成的机器,建立一定直接、感性的认识。

这点我想是本次大作业给我在计算机技术认知方面最大的影响了。也很感谢这门课程的设立、老师们的辛勤努力让我对CPU有了这样直观、深入内部的认识。

\paragraph{调试辅助带来的高效开发}
说起来比较神奇的是,这次开发过程中其实我并没遇到多少bug,但我却添加了许多调试辅助功能(即我在测试中介绍的”调试辅助面板“)。基本上所有关键信息都被引出显示并分类,因此我可以观察到系统各部分的状态细节,十分有助于高效的开发。

对于贴近硬件的开发工作,许多数据的含义不明显,随意取一个32位的二进制数而不说明意义,还要考虑系统存在bug的情况,就无法弄清楚这个数是什么意义,有没有错误了。这就像是在化学实验室拿起一杯无色无味的液体,没有人能一下说出这是什么物质,又是做什么用的。因此,与在做化学实验时一样地,在这些容易混淆的东西上贴好标签就显得尤为重要了。我想我的调试辅助功能就起到了这个左右,标明各个数据的状态和意义,实现高效开发。

\subsection{报告撰写}

\paragraph{模块按层次描述}
这次报告撰写中,让我印象较为深刻的一点是示例中给出的模块描述方式。模块介绍包括接口、功能、实现三方面。这个层次在我看来十分合理,能由浅至深地对模块进行完整的介绍。

首先是接口,这是每个模块最外部的特征,也是后续一切的基础。先说明接口名称、位宽、方向、意义等属性,十分易于理解,适合最开始说明。

其次是功能,这是模块外部特征的核心,也是针对模块使用者需要了解的部分,同时也是模块实现的基础。说明功能时不提及实现方式,我尽量使用数学的形式去描述模块的功能,面向使用者。

最后是实现,这是最深入的部分,需要对接口、功能有事先的了解。实现具体到模块内部的电路设计,仅供模块开发者、修改者了解。

可见,这三个层次由浅入深,前者为后者基础,需要了解的人群也逐步缩小。可谓逐步递进地对模块进行了说明,且不同需求的人群可以选择对应的部分阅读。很感谢报告要求中提出这个优秀的思路,使得我在撰写报告时十分高效,同时也学习到了这个精妙的思路。

\clearpage

\paragraph{关于排版系统\LaTeX}
这是我第一次尝试使用\LaTeX 撰写较为正式的报告,可能有些不完善的地方敬请谅解。

个人感觉\LaTeX 的优势主要在于其格式信息以纯文本形式存储的特性,使得所有格式都能对应到某个宏上,因此对格式的调整是可尝试可控的。反观word等排版软件的格式信息并不完全由用户可见,格式分散在多个菜单中,一下很难总览指定部分的格式,因此用户对格式的可控性较差。

此外,\LaTeX 还支持自定义宏,可以将某个重复性高的格式定义为宏,大幅度减少了重复内容的数量,便于修改。

同时,\LaTeX 也存在一些缺陷,如操作需要熟练度比图形化软件更高、预览速度(相比word、markdown等)较慢等。

最后,在此感谢卢雨轩同学分享\LaTeX 中文显示支持库的使用模板给我在此报告中使用,为此报告的撰写提供了很大帮助。

\subsection{鸣谢}
在此感谢教授我课程知识的朱文军老师,以及设计和布置此大作业的老师们。得益于朱老师清晰的讲解和老师们合理的作业设计才使得此作业完成的顺利,且收获了以上许多知识,加深了我对处理器原理的理解。


\end{document}